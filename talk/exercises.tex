\documentclass[12pt]{article}
\sloppy
\usepackage[letterpaper,margin=0.5in]{geometry}
\usepackage{framed}
\usepackage{ctable}
\usepackage{tabularx}
\usepackage{parskip}
\usepackage{xcolor}
\usepackage{graphicx}
\usepackage{hyperref}
\usepackage{listings}

\lstdefinestyle{basic}{
    captionpos=t,%
    basewidth=0.5em,%
    basicstyle=\small\ttfamily,%
    numberstyle=\tiny,%
    numbers=left,%
    stepnumber=1,%
    frame=single,%
    showspaces=false,%
    showstringspaces=false,%
    showtabs=false,%
    %
    keywordstyle=\color{blue},%
    identifierstyle=,%
    commentstyle=\color{gray},%
    stringstyle=\color{magenta}%
}
\lstdefinestyle{showspaces}{style=basic,showspaces=true}
\lstloadlanguages{sh,C++}
\lstset{language=sh,escapeinside={\#(}{)},style=basic,basicstyle=\small\ttfamily,emphstyle={\color{blue}}}

\definecolor{babyblue}{rgb}{0.54, 0.81, 0.94}
\definecolor{ballblue}{rgb}{0.13, 0.67, 0.8}
\definecolor{albatrosred}{HTML}{CA4436}

\newcommand{\code}[1]{\texttt{#1}}

\begin{document}

\section*{Exercise~1: Download and install the tutorial}

Open a terminal. Then, enter the following commands.

\begin{lstlisting}[language=sh,emph={ls,cd,mkdir,git}]
$ cd ~
$ mkdir alpstut
$ cd alpstut
$ git clone https://github.com/ALPSCore/Tutorial2.git
$ cd Tutorial2
$ tutorial=$PWD
$ ls -l
\end{lstlisting}

You should see a list of files and no error messages.

\section*{Exercise~2: Download/install prerequisites}
\subsection*{Ubuntu Linux}
\begin{lstlisting}[emph={sudo,apt,get}]
$ sudo apt-get install cmake
$ sudo apt-get install libhdf5-dev
$ sudo apt-get install libboost-all-dev
$ sudo apt-get install mpi-default-dev
\end{lstlisting}%$
\subsection*{Mac OS X}
\begin{lstlisting}[emph={port}]
$ port install alpscore
\end{lstlisting}%$
This will install the latest ALPSCore release (we don't need it!)
and prerequisites.

\subsection*{Check}
\begin{lstlisting}[language=sh,basicstyle=\ttfamily\small,emph={cmake,gcc,h5cc,mpicxx}]
$ cmake --version
$ #(\textcolor{blue}{g++}) --version
$ h5cc --version
$ mpicxx --version
\end{lstlisting}

There should be no error messages.

\section*{Exercise~3: Download and install ALPSCore.}
\begin{lstlisting}[emph={git,export,cmake,make,cd,mkdir}]
$ git https://github.com/ALPSCore/ALPSCore
$ cd ALPSCore
$ mkdir build
$ cd build
$ export ALPSCore_DIR=$PWD/install
$ cmake -DCMAKE_INSTALL_PREFIX=$ALPSCore_DIR ..
$ make
$ make test
$ make install
\end{lstlisting}%$

See also the list of CMake variables on page~\pageref{ref:cmakevars}.

\section*{Exercise 4: Build and run a dummy program\\ that uses ALPSCore and does nothing.}

The code is at \texttt{\color{ballblue}\$tutorial/step1\_trivial}.

CMake file online: \nolinkurl{https://git.io/alpstut2_s1_cmake}

Source file online: \nolinkurl{https://git.io/alpstut2_s1_main}

The directory online: \nolinkurl{https://git.io/alpstut2_s1}

\begin{lstlisting}[emph={cd,mkdir}]
$ cd $tutorial/step1_trivial
$ mkdir 000build
$ cd 000build
$ cmake ..
$ make
$ ./alpsdemo
\end{lstlisting}%$

\section*{Exercise 5: Build and run a program\\ that uses parameters.}

The code is at \texttt{\color{ballblue}\$tutorial/step2\_params}.

Online: \nolinkurl{https://git.io/alpstut2_s2}

\begin{enumerate}
\item Play with the different values of parameters.
\item Try to override them from the command line.
\item Change the program to make \code{--loud} parameter an integer,\\
      with value 0 meaning ``be quiet''.
\end{enumerate}

\begin{lstlisting}[emph={cd,mkdir}]
$ cd $tutorial/step2_params
$ mkdir 000build
$ cd 000build
$ cmake ..
$ make
$ ./alpsdemo
$ ./alpsdemo --help
$ ./alpsdemo ../params.ini
$ ./alpsdemo ../params.ini --count=3
....
\end{lstlisting}

\section*{Exercise 6: Build and run a trivial MC program}

The code is at \texttt{\color{ballblue}\$tutorial/step3\_trivial\_mc}.

Online: \nolinkurl{https://git.io/alpstut2_s3}

Note: the simulation code is split into 2 files.

\begin{enumerate}
\item Build and run.\\
\code{\$ ./alpsdemo --help}
\item Run with small counts:
\begin{lstlisting}[]
$ ./alpsdemo --count=2
$ ./alpsdemo --count=2 --verbose
\end{lstlisting}
\item Run with large count and small timelimit; time the execution:\\
\code{\$ time -p \emph{your\_command}}
\begin{lstlisting}[emph={time}]
$ time -p ./alpsdemo --count=10000000 --timelimit=1
\end{lstlisting}%$
\item Set large time limit and interrupt the program (via \code{Ctrl-C}).
\item Change \code{fraction\_completed()} so that \code{--count=0} would mean ``till timeout''.
\item Change the name of the \code{update()} method and see it does not compile any more.
\end{enumerate}

\section*{Exercise 7: Compute $\pi$ by Markov chain MC}

The code is at \texttt{\color{ballblue}\$tutorial/step4\_pi}.

Online: \nolinkurl{https://git.io/alpstut2_s4}

\begin{enumerate}
\item Build and run the program.
\item Run with various time limits:
  \begin{lstlisting}
$ ./alpsdemo --timelimit=5
$ ./alpsdemo --timelimit=10
  \end{lstlisting}
\item Run with different step sizes, compare autocorrelation lengths.
  \begin{lstlisting}
$ ./alpsdemo --step=1 
$ ./alpsdemo --step=10
$ ./alpsdemo --step=0.1
  \end{lstlisting} %$
\item Replace \code{FullBinningAccumulator} to 
\code{NoBinningAccumulator}.
\item Run with low or high step length and see the underestimated error~bars.
  \begin{lstlisting}
$ ./alpsdemo --step=1
$ ./alpsdemo --step=0.01
$ ./alpsdemo --step=10
  \end{lstlisting} %$
\end{enumerate}


\section*{Exercise 8: Running and resuming}

The code is in \texttt{\color{ballblue}\$tutorial/step5\_pi\_checkpoint}.

Online: \nolinkurl{https://git.io/alpstut2_s5}

\begin{enumerate}
\item Build the code.
\item Run the code. There is an error: find and fix it!
\item Build and run the corrected code (note more options available!).
  \begin{lstlisting}
$ ./alpsdemo --help
$ ./alpsdemo --step 1 --timelimit 5
  \end{lstlisting}
\item Note new files appear:
  \begin{itemize}
  \item  ``\code{*.out}'' file contains simulation results.
  \item  ``\code{*.clone.h5}'' file contains checkpoint.
  \end{itemize}
\item Restore the checkpoint and run for 10 more seconds:
  \begin{lstlisting}
$ ./alpsdemo alpsdemo.clone.h5 --timelimit 10 
\end{lstlisting} %$
Note:
  \begin{itemize}
  \item compulsory \code{--step} is read from the checkpoint.
  \item parameters can be overridden (like \code{--timelimit}).
  \end{itemize}
\end{enumerate}


\section*{Exercise 9: Parallel runs}

The code is in \texttt{\color{ballblue}\$tutorial/step6\_pi\_mpi}.

Online: \nolinkurl{https://git.io/alpstut2_s6}

\begin{enumerate}
\item Build the MPI-parallelized program.
\item Do timed runs with different number of processes:
  \begin{lstlisting}[emph={mpiexec}]
$ time -p mpiexec -n 1 ./alpsdemo --step=1 --timelimit=10
$ time -p mpiexec -n 2 ./alpsdemo --step=1 --timelimit=10
  \end{lstlisting} %$
\item Observe checkpoint names.
\item Try to restore from checkpoints, see how statistics builds up.
  \begin{lstlisting}[emph={mpiexec}]
$ mpiexec -n 2 ./alpsdemo alpsdemo.clone.h5
\end{lstlisting} %$
\end{enumerate}

\pagebreak
\section*{Exercise 10: Parallelize the 2D Ising code}

The code is in \texttt{\color{ballblue}\$tutorial/step7\_ising}.

Online: \nolinkurl{https://git.io/alpstut2_s7}

Steps: 
\begin{enumerate}
\item Initialize MPI environment.
\item Use \code{alps::mcmpiadapter} template.
\item Use the parallel parameter constructor.
\item Make sure each rank has its own checkpoint file.
\item Make sure only the master process outputs the results.
\end{enumerate}


\pagebreak
\section*{Appendix: Git cheat-sheet for today}
\begin{flushleft}
\textbf{Clone (``download'') repository:}\\
\lstinline[style=showspaces]|git clone |\emph{repo}%
\lstinline[style=showspaces]| |%
\emph{local\_directory}\\
or \\
\lstinline[style=showspaces]|git clone |\emph{repo}\\
Example: \\
\lstinline[style=showspaces]|git clone https://github.com/ALPSCore/Tutorial2.git|

\textbf{See the current changes:}\\
\lstinline[style=showspaces]{git diff}

\textbf{Revert changes (CAUTION --- no way back!):}\\
\lstinline[style=showspaces]|git reset --hard|
\end{flushleft}

\section*{Appendix: Important links}
\begin{flushleft}
ALPSCore site: \nolinkurl{http://alpscore.org}

ALPSCore wiki: \nolinkurl{https://github.com/ALPSCore/ALPSCore/wiki}

ALPSCore repo: \nolinkurl{https://github.com/ALPSCore/ALPSCore}

Doxygen docs: \nolinkurl{https://alpscore.ci.cloudbees.com}
              \nolinkurl{/job/alpscore_fedora17_doc-only/ALPSCore_reference/}

Tutorial repo: \nolinkurl{https://github.com/ALPSCore/Tutorial2}

Exercise solutions:
\begin{itemize}
\item Online: \nolinkurl{https://git.io/alpstut2_solutions}
\item Locally: \code{\color{ballblue}\$tutorial/X\_solutions}
\end{itemize}

\end{flushleft}

\pagebreak
\section*{Appendix: ALPSCore CMake and environment variables}
\label{ref:cmakevars}%
CMake variables: set as \code{cmake
  -D\textit{variable}=\textit{value}}.

{\small
\begin{tabularx}{\textwidth}{lcX}
  \textbf{Variable} & \textbf{Default value} & {\hfil\textbf{Comment}\hfil}\\
  \toprule
  \code{CMAKE\_CXX\_COMPILER} & (system default) & {Path to C++ compiler executable.
    Can be set only once.} \\\midrule
  \code{CMAKE\_INSTALL\_PREFIX} & \code{/usr/local} & ALPSCore target install directory. \\\midrule
  \code{CMAKE\_BUILD\_TYPE} &  &  {Specifies build type;
    set to \code{Release} to maximize performance.} \\\midrule
  \code{BOOST\_ROOT} &   & {Boost install directory.
    Set if CMake fails to find Boost.} \\\midrule
  \code{Boost\_NO\_SYSTEM\_PATHS} & false & {Disable search in default system directories.
    Set if the wrong version of Boost is found.} \\\midrule
  \code{Boost\_NO\_BOOST\_CMAKE} & false & {Disable search for Boost CMake file.
    Set if the wrong version of Boost is found.} \\\midrule
  \code{Documentation} & ON & Build ALPSCore developer's documentation. \\\midrule
  \code{ENABLE\_MPI} & ON & Enable MPI build. \\\midrule
  \code{Testing} & ON & Build unit tests (recommended). \\\midrule
  \code{ALPS\_BUILD\_SHARED} & ON & {Build ALPSCore as shared libraries.
    Mutually exclusive with \code{ALPS\_BUILD\_STATIC=ON}.} \\\midrule
  \code{ALPS\_BUILD\_STATIC} & OFF & {Build ALPSCore as static libraries.
    Mutually exclusive with \code{ALPS\_BUILD\_SHARED=ON}.} \\
  \bottomrule
\end{tabularx}
}

\vspace{2\baselineskip}
The environment variables are set via:\\
\code{\$ export \textit{variable}=\textit{value}}\\
before running CMake. The relevant
variables are:

{\small
  \begin{tabularx}{1\textwidth}{l@{\hspace{5em}}X}
    \textbf{Variable} & {\hfil\textbf{Comment}\hfil}\\
    \toprule
    \code{CXX} &  Path to C++ compiler executable.
                 Can be set only once. \\\midrule
    \code{BOOST\_ROOT} &  Boost install directory.
                        Set if CMake fails to find Boost. \\\midrule
    \code{HDF5\_ROOT}  &  HDF5 install directory.
                        Set if CMake fails to find HDF5.\\

    \bottomrule
  \end{tabularx}
}
\end{document}

